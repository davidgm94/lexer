 
\documentclass[12pt]{article}
\usepackage{hyperref}
\usepackage[margin=2.0cm]{geometry}
\usepackage{listings}
\lstset{
	basicstyle=\small\ttfamily,
	columns=flexible
	breaklines=true
}
\title{\vspace{-2.5cm}\textbf{Exploiting Modern CPU Architectures for Lexical Analysis}}
\date{\vspace{-5ex}}

\begin{document}
	\maketitle{\vspace{-1.5cm}}
	\newpage
	\tableofcontents
	\newpage
	\section{Introduction}
	TODO: does this belong here or in the next subsection?
	
	\paragraph{}As a software engineer, it does not take much to discover how fast modern computers are (from CPU internals to disk).
	
	\paragraph{}Modern server CPUs are a clear example of this. AMD provides bandwidth and frequency information about their professional CPUs in their website. At the moment of writing this, their top CPU for servers was the AMD EPYC 9654\cite{epyc}, launched in November 10, 2022. According to the official data, this is a processor with 96 cores, 192 threads, with a base clock of 2.4 GHz and all-core boost clock of 3.55 GHz; it has a 384 MB L3 Cache and it delivers a per-socket RAM bandwidth of 460.8 GB/s with up to 12 memory channels and 128 PCI-Express 5.0 slots. It is clearly a wonder of modern technology.
	
	\paragraph{}However, desktop computers are not that far behind. Likewise, Intel provides the same information for their consumer processors. With 8 performance cores and 16 efficient ones, making a total of 24 cores and 32 threads, with a base frequency of 3.2 GHz and 2.4 GHz and a turbo frequency of 5.6 GHz and 4.4 GHz, respectively, 32 MB L2 Cache plus 36 MB L3 Cache and a maximum RAM bandwidth of 89.6 GB/s, the 14th generation top-of-the-line Core i9 14900K launched recently in October 17, 2023\cite{13900k}, shows impressive hardware as well.
	
	\paragraph{}Even disk, historically considered as the slowest component of a computer (and, hence, the bottleneck), can deliver an astonishing amount of bandwidth in modern days. Just as an example, the Crucial T700, a PCIe Gen 5 NVMe SSD, can sequentially read 11,700 MB/s and write 9,500 MB/s\cite{crucialt700}.
	
	\paragraph{}Nevertheless, software speeds, in general, could not be more distant.\linebreak
	
	\begin{itemize}
		\item TODO Talk about the web being paradigm of slow software
		\item TODO Quote these resources while talking about web slowness:
		\begin{itemize}
			\item https://www.forbes.com/sites/kalevleetaru/2016/02/06/why-the-web-is-so-slow-and-what-it-tells-us-about-the-future-of-online-journalism/?sh=125e3ace72f4 
			\item https://arxiv.org/abs/1603.02293
			
			\item https://www.nngroup.com/articles/the-need-for-speed/
			\item https://2018.perfmattersconf.com/talks/
			\item https://www.manning.com/books/web-performance-in-action
		\end{itemize}
		\item TODO Talk about it not just being the web
		\item TODO mention the collapse of civilization talk and similar ones \linebreak https://www.youtube.com/watch?v=ZSRHeXYDLko
		\item TODO Talk about slowness in "high-performance" software. Quote examples (Tensorflow, compilers, windows/macos update times, software installation times windows/macos, Visual Studio (Code), Photoshop)
	
	\end{itemize}
	
	
	\subsection{Context and Rationale}
		\begin{itemize}
		\item TODO Talk about the importance of compilers and the waste of them being one of the slowest pieces of system software. Quote examples of compilers being slow: Unreal Engine, Linux kernel repository, LLVM, Chrome, etc.
		
	
	\end{itemize}
	\paragraph{}Since compilers are so widely used on a daily basis, optimizing them to the point you have some gains in speed, no matter how small it might be, you would improve greatly the experience and productivity of millions developers all over the world as well as a lot of companies and other economic entities' budget.
	\paragraph{}Due to the limited scope of this essay, both in time and workforce, the cover area must be narrowed down. Given the traditional structure of a compiler, this is divided by a set of stages: lexical analysis (lexer), syntactic analysis (parser), semantic analysis, intermediate representation (with its optional optimization stage) and code generation.
	\paragraph{}Recently some research has been exploring the possibility to take advantage of the architecture of modern CPUs to speed up lexical and syntactic analysis. Works like the papers published by professor Daniel Lemire and others (TODO quote) have brought SIMD (Single Instruction, Multiple Data) and branchless programming into scene when approaching these traditional compiler stages. The benefits of fully utilizing the underlying hardware pays off: an enormous speed up was obtained with these techniques.
	\paragraph{}Since lexical analysis receives a given immovable input format and it only loops over source file characters and outputs a list of tokens, I thought it was appropriate to stick with this compiler stage.
	TODO need to contrast if these numbers are actually correct
	\paragraph{}Some benchmark was conducted. Given that a Skylake CPU like the one being used in the test (Core i7 6700K) has nearly 40 GB/s of maximum theoretical memory bandwidth (TODO quote). Note that this is an ideal figure given by Intel, but given this data, a figure in the order of GB/s could be considered fast. I wrote a simple lexer (TODO quote) in a scalar manner and some benchmarking utilities. After 3 runs, each one with 256 MiB worth of random lexer input data, the lexer was able to run at approximately 450 MiB/s and spend about 43 ns per token:
	
	\paragraph{}TODO format properly, maybe display an image/table instead
\begin{lstlisting}
~/dev/lexer| > | zig build run -Doptimize=ReleaseFast -- 0x10000000                                                                                                                      
Running in ReleaseFast optimization mode. Preparing 0x10000000 bytes (256 MiB) worth of data
Data prepared. Running benchmark...
13343026 tokens, 574127210 ns (445.89 MiB/s, 43.03 ns/token)
~/dev/lexer| > | zig build run -Doptimize=ReleaseFast -- 0x10000000                                                                                                                      
Running in ReleaseFast optimization mode. Preparing 0x10000000 bytes (256 MiB) worth of data
Data prepared. Running benchmark...
13333143 tokens, 572905113 ns (446.85 MiB/s, 42.97 ns/token)
~/dev/lexer| > | zig build run -Doptimize=ReleaseFast -- 0x10000000                                                                                                                      
Running in ReleaseFast optimization mode. Preparing 0x10000000 bytes (256 MiB) worth of data
Data prepared. Running benchmark...
13334230 tokens, 574440131 ns (445.65 MiB/s, 43.08 ns/token)
\end{lstlisting}
	\paragraph{}Therefore one would conclude that there is still a wide margin of improvement which can be work upon.

	\subsection{Goals}
	\paragraph{}The goal of this work is to explore SIMD and branchless programming, and, in general, taking advantage of the underlying hardware (in particular, Intel Skylake or above processors) for lexical analysis.
	Specific goals are:
	\begin{itemize}
		\item Get advantage of hardware facilities to speed up commonly used code.
		\item Get as close as possible to the bandwidth limit.
		\item Profile the code and the data with the most adequate tools at disposal.
		\item Learn about the underlying hardware and its ways.
	\end{itemize}

	\subsection{Impact on Sustainability and Diversity and Social-Ethical Influence}
		TODO talk about:
		\begin{itemize}
			\item How performance gains can help reduce energy consumption and carbon footprint (Sustainability)
			\item How performance gains can help reduce costs on servers and computers in general (less resources needed to do the same job). Ethical aspect?
			\item Diversity? How is this related?
		\end{itemize}
	\subsection{Used Method and Approach}
	TODO
	\subsection{Planning}
	When it comes to the code, there are two clear jobs to be done:
	\begin{itemize}
		\item Detect tokens branchlessly and using vector instructions
		\item Profile the code to gain insight on how well the program is utilizing the CPU and explore covering unused execution ports, preventing branches and cache misses, eliminating dependency chains, etc.
	\end{itemize}
	Aside from this, a ton of literature must be read to gain more insight into how to achieve this goal and how others have faced a similar task in available papers.
	For that, I would consider resolving the first code task by mid November and focus on how to get the closest possible to that maximum CPU bandwidth from then on, literature being read at all times, but especially at the beginning.
	\subsection{Result Preview}
	TODO
	\subsection{Thesis Overview}
	TODO
	
	
	
	
	
	\section{Resources and Approach}
	TODO
	\section{Results}
	TODO
	\section{Conclusion and Future Work}
	TODO
%	\section{Glossary}
%	TODO

	\begin{thebibliography}{9}
		\bibitem{epyc}
		AMD Epyc 9654 https://www.amd.com/en/product/12191. Retrieved 24 October 2023.
		\bibitem{13900k}
		Intel Core i9 13900K https://www.intel.com/content/www/us/en/products/sku/236773/intel-core-i9-processor-14900k-36m-cache-up-to-6-00-ghz/specifications.html Retrieved 24 October 2023.
		\bibitem{crucialt700}
		Crucial T700 https://www.crucial.com/ssd/t700/CT1000T700SSD5.html Retrieved 24 October 2023.
	\end{thebibliography}
	TODO
\end{document}
